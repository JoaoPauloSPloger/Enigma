\documentclass[12pt, a4paper]{article}

% ==== Codificação, Idioma e Tipografia ====
\usepackage[utf8]{inputenc}
\usepackage[T1]{fontenc}
\usepackage[brazil]{babel}
\usepackage{microtype} % Melhora o espaçamento e justificação
\usepackage{lmodern}   % Fonte moderna vetorial

% ==== Matemática e Símbolos ====
\usepackage{amsmath, amssymb, amsfonts, amsthm}
\usepackage{mathtools}

% ==== Layout e Estilo ====
\usepackage{geometry}
\geometry{
    a4paper,
    left=30mm,
    right=20mm,
    top=30mm,
    bottom=20mm
}
\usepackage{titlesec}
\usepackage{enumitem}
\usepackage{fancyhdr}
\usepackage{graphicx}
\usepackage{xcolor}
\usepackage{booktabs} % Tabelas profissionais
\usepackage[ruled,vlined]{algorithm2e} % Pseudocódigo

% ==== Hiperlinks e URLs ====
\usepackage[colorlinks=true, linkcolor=blue, citecolor=green, urlcolor=blue]{hyperref}
\usepackage{url}

% ==== Configuração de Seções ====
\titleformat{\section}{\large\bfseries\uppercase}{\thesection}{1em}{}
\titleformat{\subsection}{\bfseries}{\thesubsection}{1em}{}

% ==== Cabeçalho e Rodapé ====
\pagestyle{fancy}
\fancyhf{}
\rhead{\footnotesize \textit{Enigma Avançada - Criptografia Dinâmica}}
\lhead{\footnotesize \textbf{JP(TY)SP}}
\cfoot{\thepage}
\renewcommand{\headrulewidth}{0.4pt}

% ==== Metadados ====
\title{
    \vspace{-2cm}
    \rule{\linewidth}{0.5mm} \\[0.4cm]
    \Huge \textbf{Enigma Avançada} \\
    \Large Uma Abordagem Dinâmica e Polialfabética \\ Inspirada na Máquina Enigma \\[0.4cm]
    \rule{\linewidth}{0.5mm}
}
\author{\textbf{JP(TY)SP}}
\date{\today}

\begin{document}

\maketitle

\begin{abstract}
    \noindent Este artigo apresenta o projeto \textit{Enigma Avançada}, um sistema de criptografia digital inspirado na histórica máquina eletromecânica utilizada durante a Segunda Guerra Mundial. Proposto como uma ferramenta educacional e experimental, o sistema moderniza o conceito de rotores, aplicando-o sobre um alfabeto Base32 (RFC 4648) e introduzindo vetores de deslocamento baseados em funções matemáticas não lineares. O trabalho detalha a arquitetura do sistema, as equações de deslocamento rotacional, a análise do espaço de chaves e a implementação de uma interface interativa. Adicionalmente, discute-se a inclusão de camadas de ofuscação visual através de alfabetos simbólicos, visando explorar conceitos de esteganografia e representação estética de dados cifrados.
    
    \vspace{0.5cm}
    \noindent \textbf{Palavras-chave:} Criptografia Polialfabética; Máquina Enigma; Base32; Cifras de Substituição Dinâmica; Segurança Computacional.
\end{abstract}

\vspace{0.5cm}

\begin{center}
    \textbf{Keywords:}
    {\small\itshape
    Polyalphabetic Cryptography, Enigma Machine, Advanced Cipher System, Digital Rotors, Base32 Encoding, Dynamic Substitution Cipher, Cryptographic Engineering.
    }
\end{center}

\newpage
\tableofcontents
\newpage

\section{Introdução}

A Máquina Enigma, patenteada originalmente em 1918 pelo engenheiro alemão Arthur Scherbius, representa um marco na história da criptografia. Inicialmente destinada ao uso comercial, foi amplamente adotada pela Wehrmacht e pela Kriegsmarine durante a Segunda Guerra Mundial, tornando-se peça central nas estratégias de comunicação do Eixo.

O funcionamento da Enigma baseava-se em um circuito eletromecânico complexo. Ao pressionar uma tecla, um sinal elétrico percorria um \textit{plugboard} (painel de conexões), atravessava uma série de três a cinco rotores móveis e era refletido por um componente denominado \textit{Umkehrwalze} (refletor), retornando pelos rotores em caminho reverso até iluminar o caractere cifrado. A cada tecla pressionada, pelo menos um rotor avançava, alterando o circuito elétrico e, consequentemente, o alfabeto de substituição. Essa propriedade polialfabética garantia que a mesma letra, digitada repetidamente, fosse cifrada em caracteres distintos, elevando exponencialmente a complexidade da criptoanálise para os padrões da época \cite{wikiEnigma, wikiEnigmaPTBR}.

A quebra da Enigma, iniciada por matemáticos poloneses como Marian Rejewski e consolidada pelos esforços britânicos em Bletchley Park sob a liderança de Alan Turing, não apenas encurtou a guerra em anos, mas também impulsionou o desenvolvimento da computação moderna \cite{TNMOCenigma}.

\subsection{Objetivo do Projeto}

O projeto \textit{Enigma Avançada} propõe uma releitura digital deste mecanismo clássico, expandindo-o para atender a requisitos didáticos modernos. Diferente da implementação física restrita a 26 caracteres do alfabeto latino, esta versão adota padrões digitais robustos e flexibilidade matemática.

Os objetivos específicos incluem:
\begin{enumerate}
    \item \textbf{Expansão do Mecanismo Rotacional:} Implementação de um vetor de até 10 rotores virtuais independentes $\vec{R} = [R_1, \dots, R_{10}]$, superando o limite mecânico tradicional.
    \item \textbf{Codificação Base32:} Adoção do padrão RFC 4648 (Alfabeto $A-Z$ e $2-7$) para a entrada de dados, garantindo que o texto cifrado seja seguro para transmissão em URLs e sistemas de arquivos, eliminando ambiguidade de caixa (case-insensitive).
    \item \textbf{Funções de Deslocamento Não Lineares:} Substituição da fiação física fixa por funções matemáticas dinâmicas (ex.: somas quadráticas, produtos e composições modulares) para determinar o deslocamento de cifra.
    \item \textbf{Interface Interativa:} Desenvolvimento de uma interface web responsiva para visualização em tempo real do estado interno da máquina, promovendo o aprendizado sobre entropia e ciclos criptográficos.
\end{enumerate}

\section{Fundamentação Teórica}

\subsection{Cifras Polialfabéticas e Estado}
Uma cifra de substituição monoalfabética mapeia cada símbolo do texto claro a um símbolo fixo do texto cifrado ($P \to C$). Já sistemas polialfabéticos, como a Enigma, utilizam uma função $E(P, S_t) \to C$, onde $S_t$ é o estado interno da máquina no tempo $t$. Como $S_t$ muda a cada iteração ($S_{t+1} = f(S_t)$), a relação entre texto claro e cifrado é dinâmica, resistindo a análises de frequência simples.

\subsection{Codificação Base32}
A codificação Base32 utiliza um alfabeto de 32 caracteres, consistindo nas letras maiúsculas A--Z e nos dígitos 2--7. Sendo $2^5 = 32$, cada caractere pode representar exatamente 5 bits de informação. A escolha deste alfabeto para a \textit{Enigma Avançada} visa:
\begin{itemize}
    \item \textbf{Compatibilidade:} Caracteres seguros para URLs e nomes de arquivos.
    \item \textbf{Uniformidade:} O espaço de módulo $m=32$ simplifica operações bit a bit, diferentemente do módulo 26.
\end{itemize}

\section{Arquitetura do Sistema}

A arquitetura da \textit{Enigma Avançada} é composta por três estágios principais: Pré-processamento, Núcleo Criptográfico e Pós-processamento.

\subsection{Definição Formal}

Seja $\mathcal{A}$ o alfabeto Base32 tal que $|\mathcal{A}| = 32$. Seja $P$ o texto claro e $C$ o texto cifrado. O estado da máquina é definido pelo vetor de rotores $\vec{R} \in \mathbb{Z}_{26}^{10}$ (note que os rotores simulam a mecânica original de 26 posições, embora o deslocamento final seja aplicado $\mod 32$).

\subsection{Equações de Deslocamento ($\Psi$)}

O "coração" da cifra é a função de deslocamento $\Psi(\vec{R})$, que determina o valor $S$ a ser somado ao índice do caractere atual. O sistema implementa múltiplos modos de operação:

\begin{description}
    \item[Modo 1: Soma Linear] $S = \sum R_i$
    \item[Modo 2: Quadrática] $S = \sum R_i^2$
    \item[Modo 3: Cúbica] $S = \sum R_i^3$
    \item[Modo 4: Produto Ponderado] $S = \prod (R_i + 1)$
    \item[Modo 5: Ponderação por Índice] $S = \sum (i \cdot R_i)$
    \item[Modo 8: Super Enigma (Composta)]
    Este modo combina todas as estratégias anteriores para maximizar a entropia do deslocamento. Definimos os componentes auxiliares:
    \[ A = \sum R_i, \quad B = \sum R_i^2, \quad C = \sum R_i^3 \]
    \[ D = \prod (R_i + 1), \quad E = \sum i \cdot R_i, \quad F = A^2 + B \]
    \[ G = \sum_{k=0}^{4} (R_{2k} \cdot R_{2k+1}) \]
    
    O deslocamento final $S$ é dado por uma função de agregação não linear sobre o conjunto $T = \{A, B, C, D, E, F, G\}$:
    \begin{equation}
        S = \left( \sum_{X \in T} X \cdot \prod_{Y \in T} (Y - 1) \right)
    \end{equation}
    Este valor é calculado com precisão arbitrária e posteriormente reduzido modulo 32 para a operação de cifra.
\end{description}

\section{Algoritmo de Cifra}

O processo de cifragem e decifragem para um caractere $x$ na posição $t$ do texto é descrito abaixo.

\subsection{Cifragem}
Dado um caractere de entrada $x$, obtemos seu índice $idx(x) \in [0, 31]$. O índice cifrado $c$ é:
\begin{equation}
    c = (idx(x) + \Psi(\vec{R}_t)) \mod 32
\end{equation}
Onde $\vec{R}_t$ é a configuração dos rotores no tempo $t$.

\subsection{Decifragem}
O processo inverso subtrai o deslocamento gerado pelo mesmo estado $\vec{R}_t$:
\begin{equation}
    p = (idx(c) - \Psi(\vec{R}_t)) \mod 32
\end{equation}
\textit{Nota:} Em aritmética modular, subtrair $k$ equivale a somar $m-k$. Portanto, $p = (idx(c) - S + 32) \mod 32$.

\subsection{Atualização de Estado (Stepping)}
Após o processamento de cada caractere, o vetor de rotores é atualizado. A implementação simula um odômetro digital: o rotor $R_{10}$ incrementa a cada passo. Quando $R_i$ completa uma volta ($25 \to 0$), o rotor vizinho $R_{i-1}$ é incrementado.

\begin{algorithm}[H]
\SetAlgoLined
\KwData{Vetor de Rotores $\vec{R}$}
\KwResult{Vetor $\vec{R}$ atualizado}
 $i \leftarrow 10$\;
 \While{$i \geq 1$}{
  $R_i \leftarrow (R_i + 1) \mod 26$\;
  \eIf{$R_i \neq 0$}{
   \textbf{break} (fim do carry)\;
   }{
   $i \leftarrow i - 1$\;
  }
 }
 \caption{Lógica de Atualização dos Rotores (Odômetro)}
\end{algorithm}

\section{Análise de Segurança e Espaço de Chaves}

A robustez da \textit{Enigma Avançada} reside no tamanho do seu espaço de estados e na não-linearidade das funções de deslocamento.

\subsection{Espaço de Chaves ($\mathcal{K}$)}
O espaço total de configurações iniciais é determinado pelos 10 rotores de 26 posições:
\[ |\mathcal{K}_{rotores}| = 26^{10} \approx 1,41 \times 10^{14} \]
Considerando os 8 modos de operação matemática selecionáveis, o espaço total expande para:
\[ |\mathcal{K}_{total}| = 8 \times 26^{10} \approx 1,13 \times 10^{15} \]
Isso equivale a aproximadamente 50 bits de entropia apenas na configuração inicial, sem considerar a complexidade criptoanalítica introduzida pelas equações não lineares (Modo Super Enigma), que dificultam ataques de texto claro conhecido.

\section{Implementação e Interface}

O sistema foi desenvolvido utilizando tecnologias web padrão (HTML5, CSS3 e JavaScript ES6+), permitindo execução client-side segura (sem envio de dados para servidores).

\begin{itemize}
    \item \textbf{Visualização:} Os rotores são exibidos como componentes gráficos dinâmicos.
    \item \textbf{Alfabeto Galáctico:} Uma camada opcional de ofuscação visual mapeia os caracteres Base32 resultantes para glifos simbólicos (inspirados no \textit{Standard Galactic Alphabet}). Esta etapa não adiciona entropia matemática, mas serve como barreira visual e elemento lúdico.
\end{itemize}

\section{Conclusão}

O projeto \textit{Enigma Avançada} demonstra como princípios criptográficos históricos podem ser revitalizados com técnicas modernas de engenharia de software. Ao desacoplar a lógica de cifra das limitações mecânicas, criou-se um sistema com alto valor pedagógico, permitindo a exploração prática de conceitos como aritmética modular, entropia de Shannon e codificação de dados. Futuras iterações poderão incluir a integração de um \textit{Plugboard} digital e a capacidade de processamento de arquivos binários.

% ==== Bibliografia ====
\newpage
\begin{thebibliography}{9}

\bibitem{wikiEnigma}
WIKIPEDIA. \textit{Enigma machine}. Disponível em: \url{https://en.m.wikipedia.org/wiki/Enigma_machine}. Acesso em: 04 ago. 2025.

\bibitem{wikiEnigmaPTBR}
WIKIPEDIA. \textit{Enigma (máquina)}. Disponível em: \url{https://pt.wikipedia.org/wiki/Enigma_(m%C3%A1quina)}. Acesso em: 04 ago. 2025.

\bibitem{TNMOCenigma}
THE NATIONAL MUSEUM OF COMPUTING. \textit{Bombe and Enigma: The History}. Disponível em: \url{https://www.tnmoc.org/bh-2-the-enigma-machine}. Acesso em: 04 ago. 2025.

\bibitem{rfc4648}
JOSEFSSON, S. \textit{The Base16, Base32, and Base64 Data Encodings}. RFC 4648, Internet Engineering Task Force, 2006.

\end{thebibliography}

\end{document}
